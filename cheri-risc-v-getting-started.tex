\documentclass[12pt,twoside,openright,a4paper]{article}
\setlength{\oddsidemargin}{-0.4mm} % 25 mm left margin
\setlength{\evensidemargin}{\oddsidemargin}
\setlength{\textwidth}{160mm}      % 25 mm right margin
\setlength{\topmargin}{-5.4mm}     % 20 mm top margin
\setlength{\headheight}{5mm}
\setlength{\headsep}{5mm}
\setlength{\footskip}{10mm}
\setlength{\textheight}{237mm}     % 20 mm bottom margin

%\documentclass[11pt]{article}
%\usepackage[UKenglish]{isodate}%UK date endian
\usepackage[headings]{fullpage}
\usepackage[hidelinks]{hyperref}

\usepackage{bytefield}
\usepackage{color}
\usepackage[scaled=0.8]{DejaVuSansMono}
\usepackage[T1]{fontenc}
\usepackage{listings}
\usepackage{subcaption}
\usepackage{times}
\usepackage{url}
\usepackage[svgnames]{xcolor}
\definecolor{lightgray}{gray}{0.8}
\usepackage{xspace}
\usepackage{xfrac}

\usepackage{cleveref}

\renewcommand{\UrlFont}{\ttfamily\small}

\newcommand{\baselineboxformatting}[1]{%
  % Measure size of contents
  \sbox0{#1}%
  % Use the difference between the contents' height and the bitbox's height,
  % clamped to [-.44\baselineskip, 0], as our minimum depth.
  \setlength{\skip0}{\ht0 - \height}%
  \ifdim\skip0>0pt%
    \setlength{\skip0}{0}%
  \else%
    \ifdim\skip0<-.44\baselineskip%
      \setlength{\skip0}{-.44\baselineskip}%
    \fi%
  \fi%
  \centering\rule[\skip0]{0pt}{\height}#1}
\bytefieldsetup{boxformatting=\baselineboxformatting}

\lstset{basicstyle=\footnotesize\ttfamily}
%\newcommand{\ccode}[1]{\lstinline[language={C}]{#1}}
%\newcommand{\cxxcode}[1]{\lstinline[language={C++}]{#1}}
\newcommand{\ccode}[1]{{\small\ttfamily{#1}}}
\newcommand{\cxxcode}[1]{{\ccode{#1}}}
\newcommand{\cconst}[1]{{\ccode{#1}}}
\newcommand{\cfunc}[1]{{\ccode{#1()}}}
\newcommand{\cvar}[1]{{\ccode{#1}}}
\newcommand{\pathname}[1]{{\ccode{#1}}}
\newcommand{\commandline}[1]{{\ccode{#1}}}

\newcommand{\ptrdifft}{{\ccode{ptrdiff\_t}}\xspace}
\newcommand{\sizet}{{\ccode{size\_t}}\xspace}
\newcommand{\ssizet}{{\ccode{ssize\_t}}\xspace}
\newcommand{\vaddrt}{{\ccode{vaddr\_t}}\xspace}
\newcommand{\cuintptrt}{{\ccode{uintptr\_t}}\xspace}
\newcommand{\cintptrt}{{\ccode{intptr\_t}}\xspace}
\newcommand{\ccharstar}{{\ccode{char *}}\xspace}
\newcommand{\cvoidstar}{{\ccode{void *}}\xspace}
\newcommand{\clongt}{{\ccode{long}}\xspace}
\newcommand{\cintt}{{\ccode{int}}\xspace}
\newcommand{\cintttt}{{\ccode{int32\_t}}\xspace}
\newcommand{\cintsft}{{\ccode{int64\_t}}\xspace}

\newcommand{\SIGPROT}{{\ccode{SIGPROT}}\xspace}

\newcommand{\note}[2]{{\color{blue}[ Note: #1 - #2]}}
\renewcommand{\note}[2]{\relax\ifhmode\unskip\fi}
\newcommand{\arnote}[1]{\note{#1}{Alex R.}}
\newcommand{\bdnote}[1]{\note{#1}{Brooks D.}}
\newcommand{\rwnote}[1]{\note{#1}{Robert W.}}
\newcommand{\amnote}[1]{\note{#1}{Alfredo M.}}
\newcommand{\psnote}[1]{\note{#1}{Peter S.}}
\newcommand{\pgnnote}[1]{\note{#1}{Peter N.}}
\newcommand{\jrtcnote}[1]{\note{#1}{Jess C.}}

\hyphenation{Free-BSD}
\hyphenation{Free-RTOS}
\hyphenation{Cheri-BSD}
\hyphenation{Cheri-Free-RTOS}
\hyphenation{Cheri-ABI}
\hyphenation{Web-Kit}

\title{Getting Started with CHERI-RISC-V (DRAFT)}
\author{Robert N. M. Watson and Sundry Other Authors and Contributors}

\begin{document}
\sloppy

%% CL tech-report format provides its own cover page.  Comment for final
%% version.
\maketitle

%% CL tech-report format requires page numbering to start at 3.  Uncomment for
%% final version.
%\setcounter{page}{3}
%%

%
% Keep Abstract in sync with the Introduction.
%
\begin{abstract}
This document is a brief introduction to the CHERI-RISC-V ISA, toolchain,
QEMU emulator, and software stack.
It is intended as a starting point for those wishing to experiment with
CHERI-RISC-V, and complements other reports such as the Introduction to CHERI,
the CHERI ISA Specification, and the CHERI Pure-Capability C/C++ Programming
Guide.
\end{abstract}

\newpage
\setcounter{tocdepth}{2}
\tableofcontents

\newpage

\section{Introduction}

%
% Keep Abstract in sync with the Introduction.
%
This document is a brief introduction to the CHERI-RISC-V ISA, toolchain,
QEMU emulator, and software stack.
It is intended as a starting point for those wishing to experiment with
CHERI-RISC-V, and complements other reports such as the CHERI ISA
specification and CHERI C/C++ pure-capability programming guide.
This document is a brief introduction to the pure-capability CHERI C/C++
programming languages.

\section{CHERI}

CHERI extends conventional processor Instruction-Set Architectures (ISAs) with
support for \textit{architectural capabilities}.
One important use for this new hardware data type is in the implementation
of safer C/C++ pointers and the code or data they point at.
Our technical report, \textit{An Introduction to CHERI}, provides a high-level
overview of the CHERI architecture, ISA modeling, hardware implementations,
and software stack~\cite{UCAM-CL-TR-941}.

\subsection{CHERI capabilities}

\begin{figure}[b]
\hspace{2.5cm}
% Tag
\begin{subfigure}[t!]{0.1\textwidth}
\begin{bytefield}[bitwidth=3pt]{1}
% \bitheader[endianness=big]{~,~} \\
\begin{leftwordgroup}{1-bit tag}
\bitbox{1}{}
\end{leftwordgroup}
\end{bytefield}
\end{subfigure}
% Capability
\begin{subfigure}[t!]{0.1\textwidth}
\begin{bytefield}[bitwidth=3pt]{64}
\bitheader[endianness=big]{0,63} \\
\begin{rightwordgroup}{128-bit \\ in-memory \\ capability}
\bitbox{16}{perms} & \bitbox{3}{\color{lightgray}\rule{\width}{\height}} & \bitbox{15}{otype} & \bitbox{30}{bounds} \\
\bitbox[lrb]{64}{64-bit~address}
\end{rightwordgroup}
\end{bytefield}
\end{subfigure}
\caption{128-bit CHERI Concentrate capability representation: 64-bit address
  and metadata in addressable memory; and 1-bit out-of-band tag.}
\label{figure:cheri-capability-representation}
\end{figure}

CHERI capabilities are twice the width of the native integer pointer type of
the baseline architecture: there are 128-bit capabilities on 64-bit platforms,
and 64-bit capabilities on 32-bit platforms.
Each capability consists of an integer (virtual) address of the natural size for
the architecture (e.g., 32 or 64 bit), and also additional metadata that is
compressed in order to fit in the remaining 32 or 64 bits of the capability
(see \Cref{figure:cheri-capability-representation}).
In addition, they are associated with a 1-bit validity ``tag'' whose value is
maintained in registers and memory by the architecture.
Each element of the capability contributes to the protection model:

\begin{description}
\item[Validity tag] The tag tracks the validity of a capability.
  If invalid, the capability cannot be used for load, store, instruction
  fetch, or other operations.
  It is still possible to extract fields from an invalid capability,
  including its address.
  Subject to precision limits of the bounds compression model, a capability's
  address (i.e., where it points) can be changed while maintaining the
  validity tag.

\item[Bounds] The lower and upper bounds are addresses restricting the
  portion of the address space within which the capability can be used for
  load, store, and instruction fetch.

\item[Permissions] The permissions mask controls how the capability can be
  used -- for example, by providing loading and storing of data and/or
  capabilities.

\item[Object type] If not equal to $-1$, the capability is ``sealed'' and
  cannot be modified or dereferenced, but can be used to implement opaque
  pointer types.
  This feature is not described further in this document, as it is primarily
  used to implement software compartmentalization rather than object-level
  memory protection.
\end{description}

When stored in memory, valid capabilities must be naturally aligned -- i.e., at
64-bit or 128-bit boundaries, depending on capability size -- as that is the
granularity at which in-memory tags are maintained.
Partial or complete overwrites with data, rather than a complete overwrite
with a valid capability, lead to the in-memory tag being cleared, preventing
corrupted capabilities from later being dereferenced.

In order to reduce the memory footprint of capabilities, capability
compression is used to reduce the overhead of bounds so that the full
capability, including address, permissions, and bounds fits within 128
bits (plus the 1-bit out-of-band tag).
Bounds compression takes advantage of redundancy between the address
and the bounds, which occurs because a pointer typically falls within (or
close to) its associated allocation, and because allocations are typically
well aligned.
The compression scheme uses a floating-point representation, allowing high-precision bounds for small
objects, but requiring stronger alignment and padding for larger allocations.

\subsection{Architectural rules for capability use}

The architecture enforces several important security properties on changes to
this metadata:

\begin{description}
\item[Provenance validity] ensures that capabilities can be used -- for
  load, store, instruction fetch, etc., -- only if they are derived via valid
  transformations of valid capabilities.
  This property holds for capabilities in both registers and memory.

% \item[Capability integrity] prevents direct in-memory manipulation of
%   capabilities.  (Although this property is subsumed
%   under the previous property, it seems worth stating on its own.)
% \pgnnote{Does that added sentence work?}
% \rwnote{I'm not really sure that that helps.}

%  \psnote{As they are stated above, ``provenance validity'' 
%    subsumes ``capability integrity'', which is a bit confusing.  One
%    could just lose the latter, or (I suppose) split ``provenance
%    validity'' into the case of capability construction in registers,
%    via loads and register
%    operations and the case of capability (de)construction in memory,
%    via good and bad store operations}

%  \psnote{As stated above, ``provenance validity'' involves both the
%    construction and use of capabilities. Think that's ok, but a
%    different slicing of the concepts would be to have it just be
%    construction.   If sticking with ``can be used'', then somehow the
%    text should be elaborated to not forbid use in non-authorising
%    ways, e.g., ``using'' a possibly-non-valid capability by pulling
%    out its address}
  
\item[Monotonicity] requires that any capability derived from another
  cannot exceed the permissions and bounds of the capability from which it was derived.

% \psnote{That's a bit odd, as a capability is really just a pure
%   value, not a mutable thing.  A better (but still a bit fuzzy)
%   statement would be something like ``Monotonicity ensures that a
%   valid capability can only be constructed from another capability
%   with the same or greater authority''.  
% Beyond that, one should talk not just about monotonicity of capability
% construction, but monotonicity of the set of all \emph{reachable
%   capabilities} -- compare with the cheri\_formal\_paper Section IV. }
% \rwnote{This comment partially addressed.}
% \rwnote{This report does not discuss compartmentalisation at all, so is
% uninterested in the other definition of monotonicity.}
  
\end{description}

At boot time, the architecture provides initial capabilities to the firmware,
allowing data access and instruction fetch across the full address space.
Additionally, all tags are cleared in memory.
Further capabilities can then be derived (in accordance with the monotonicity
property) as they are passed from firmware to boot loader, from boot loader to
hypervisor, from hypervisor to the OS, and from the OS to the application.
At each stage in the derivation chain, bounds and permissions may be
restricted to further limit access.
For example, the OS may assign capabilities for only a limited portion of the
address space to the user software, preventing use of other portions of the
address space.

% These capabilities describe the set of memory access permissions held by each
% software component.

%The initial capabilities are then
%  derived from existing valid capabilities, in accordance with the monotonicity
%  property;
% \pgnnote{This seems cleaner.}
% \rwnote{More clear but less correct -- the initial capabilities are never
%   derived from other capabilities.}

Similarly, capabilities carry with them \textit{intentionality}: when a
process passes a capability as an argument to a system call, the OS kernel can
carefully use only that capability to ensure that it does not access other
process memory that was not intended by the user process -- even though the
kernel may in fact have permission to access the entire address space through
other capabilities it holds.
This is important as it prevents ``confused deputy'' problems, in which a more
privileged party uses an excess of privilege when acting on behalf of a less
privileged party, performing operations that were not intended to be
authorized.
For example, this prevents the kernel from overflowing the bounds on a
userspace buffer when a pointer to the buffer is passed as a
system-call argument.

These architectural properties provide the foundation on which a
capability-based OS, compiler, and runtime can implement C/C++-language memory
safety.

\section{The CHERI-RISC-V ISA}

\textbf{Brief summary with pointers to ISA spec, table of instructions,
  perhaps illustrative comparisons between some RISC-V code and some CHERI
  code.}

\section{cheribuild}

\textbf{Brief intro to cheribuild, where to get it, how to keep it up-to-date
  and why.}

\section{CHERI-QEMU}

\textbf{Building, using.}

\section{CHERI-RISC-V Clang/LLVM/LLD}

\textbf{Building, using.  Including how to compile for CheriABI on CheriBSD.}

\section{CHERI GDB}

\textbf{Building, CHERI-specific things.}

\section{CheriBSD/RISC-V}

\textbf{Building, booting, using.  Short command line to generate an image
  suitable to use with CHERI-QEMU.}

\section{CheriFreeRTOS}
CheriFreeRTOS is a version of the open-source FreeRTOS embedded operating
system that has been extended to support CHERI memory protection.
CheriFreeRTOS is compiled as a pure-capability binary, and includes
strong hardware-supported spatial memory protection and pointer protection
for the stack, heap, and global variables.\\

FreeRTOS' source code structure is divided into two main directories:

\begin{itemize}
	\item \textbf{Source:} Contains architecture-independent implementation
	 of OS services such as of threading, scheduling, memory management,
	 queues, etc. The architecture-dependent code goes into the
	 \textit{portable} sub-directory.
	\item \textbf{Demo:} Contains both application code and device drivers
	for a specific hardware board or platform. Each Demo has a Makefile
	that specifies the source files are required for it from
	\textit{Source} and \textit{Demo} directories (and optionally other
	directories as well). The final target is a single executable file.

\end{itemize}

In CheriFreeRTOS, we have ported both Source and two Demos to work
in pure-capability mode. The two Demos are:

\begin{itemize}
	\item \textbf{RISC-V-Generic:} This is a generic parameterized Demo
	 that can be built and run on simulators such as Spike, QEMU, and
	 Sail. It only has the bare minimum drivers such as console and
		 timer to run the FreeRTOS \textit{blinky} Demo.
	\item \textbf{RISC-V\_Galois\_P1:} This is a more sophisticated Demo
	that is intended to run on VCU-118 FPGA board. It has a collection
	of device drivers such as UART and Ethernet and a few FreeRTOS
	applications including \textit{netboot} and \textit{blinky}.
\end{itemize}

\subsection{Building and Running}

cheribuild has support for building CheriFreeRTOS and running it on our
CHERI-QEMU.

\begin{lstlisting}
./cheribuild.py freertos-baremetal-riscv64-purecap -d
\end{lstlisting}

This command builds CheriFreeRTOS and its compiler-rt and newlib dependecies,
all in pure-capability mode. It builds both demos discussed earlier.\\

An example command to run the \textit{RISC-V-Generic's } \textit{blinky}
demo on CHERI-QEMU is:

\begin{lstlisting}
./cheribuild.py run-freertos-baremetal-riscv64-purecap
\end{lstlisting}


\section{CHERI-RTEMS}
RTEMS is free open source software that supports multi-processor
systems for over a dozen CPU architectures and over 150 specific
system boards. In addition, RTEMS is designed to support embedded
applications with the most stringent real-time requirements while
being compatible with open standards such as POSIX. RTEMS includes
optional functional features such as TCP/IP and file systems while
still offering minimum executable sizes under 20 KB in useful
configurations.\\


There are over 500 RTEMS applications including benchmarks and tests.
Furthermore, there is a recent project called \textit{libbsd} that reuses
the FreeBSD's TCP/IP stack, drivers and applications within RTEMS. Hence,
it would be convenient to share some codebase between CHERI-RTEMS and
CheriBSD. \\

CHERI-RTEMS is a prototype of RTEMS adapted to be built and run in purecap
mode. The sample applications can run on CHERI-Spike, CHERI-QEMU, and FPGA.
The port is based on \textit{rv64imafdc} architecture.

\subsection{Building}
The easiest way to build CHERI-RTEMS is from the cheribuild project using
the following command:

\begin{lstlisting}
./cheribuild.py rtems -d
\end{lstlisting}

This command will build and install CHERI-RTEMS and all of its dependencies (if required)
such as Clang/LLVM/LLD, newlib and compiler-rt. The output of running this command is
ELF executables. Each ELF is a stand-alone bootable image of a CHERI-RTEMS
kernel and an application. A brief description of the frequently used sample
applications is as follows:
\begin{itemize}
	\item \textbf{hello}: A Hello World application with a single RTEMS task. It only needs
	a console driver and does not require timers or interrupts.
	\item \textbf{ticker}: An application that demonstrates the use of 3 RTEMS tasks being
	context-switched while using a timer to sleep and wake up. This application tests the
	most critical parts of RTEMS such as interrupt handling, context switching, and console and
	clock drivers.
	\item \textbf{capture}: A simple RTEMS shell that can be used to print information about
	the run-time system resources such as the number of tasks, drivers, semaphores and other
	RTEMS managers.
\end{itemize}

cheribuild will build and install the CHERI-RTEMS sample applications for 3 BSPs: CHERI-Spike,
CHERI-QEMU, and FPGA (GFE-P2).

\subsection{Running}

CHERI-RTEMS can run on CHERI-Spike and CHERI-QEMU. Both can be build from cheribuild:

\begin{lstlisting}
./cheribuild.py spike
./cheribuild.py qemu
\end{lstlisting}

To run on Spike:

\begin{lstlisting}[breaklines]
spike $PATH_INSTALLED_RTEMS/rv64imafdcxcheri_medany/testsuites/samples/ticker.exe
\end{lstlisting}

To run on CHERI-QEMU:

\begin{lstlisting}[breaklines]
qemu-system-riscv64cheri -nographic -M virt -bios $PATH_INSTALLED_RTEMS/rv64xcheri_qemu/testsuites/samples/ticker.exe
\end{lstlisting}

\section{CHERI-RISC-V Cores on FPGA}

\textbf{Table of cores, brief summary.  It is not the aim to describe how to
  get them running here, but it will be useful information for those using
  CHERI-RISC-V cores as part of FETT, etc.}

\section{Further reading}
\label{sec:further_reading}

\textbf{XXXRW: To update for this report.}

The primary reference for the CHERI Instruction-Set Architecture (ISA) is the
ISA specification; at the time of writing, the most recent version is CHERI
ISAv7~\cite{UCAM-CL-TR-927}:

\smallskip
\noindent
\url{https://www.cl.cam.ac.uk/techreports/UCAM-CL-TR-927.pdf}
\smallskip

\noindent
Our technical report, \textit{An Introduction to CHERI}, provides a high-level
overview of the CHERI architecture, ISA modeling, hardware implementations,
and software stack~\cite{UCAM-CL-TR-941}:

\smallskip
\noindent
\url{https://www.cl.cam.ac.uk/techreports/UCAM-CL-TR-941.pdf}
\smallskip

\noindent
We published a paper on idiomatic C and spatial memory protection at ASPLOS
2015~\cite{ChisnallCPDP11}:

\smallskip
\noindent
\url{https://www.cl.cam.ac.uk/research/security/ctsrd/pdfs/201503-asplos2015-cheri-cmachine.pdf}
\smallskip

\noindent
We published a paper on CheriABI and the adaptation of a complete OS userspace
and application suite to a pure-capability process environment at ASPLOS
2019~\cite{davis2019:cheriabi}:

\smallskip
\noindent
\url{https://www.cl.cam.ac.uk/research/security/ctsrd/pdfs/201904-asplos-cheriabi.pdf}
\smallskip

\noindent
We also released an extended technical-report version of this paper that
includes greater implementation detail~\cite{UCAM-CL-TR-932}:

\smallskip
\noindent
\url{https://www.cl.cam.ac.uk/techreports/UCAM-CL-TR-932.pdf}
\smallskip

\noindent
We published a paper on CHERI and temporal memory safety for the heap at
Oakland 2020~\cite{filardo:cornucopia}:

\smallskip
\noindent
\url{https://www.cl.cam.ac.uk/research/security/ctsrd/pdfs/2020oakland-cornucopia.pdf}
\smallskip

%\textbf{XXX: Any other pointers?}

\section{Acknowledgements}

\textbf{XXXRW: To update}

We gratefully acknowledge the helpful feedback from our colleagues, including
Peter G. Neumann, John Baldwin, Peter Sewell, Brett
Gutstein, Alfredo Mazzinghi, Simon W. Moore, Lee Smith, and Paul Gotch.
This work was supported by the Defense Advanced Research Projects Agency (DARPA) and the Air Force Research Laboratory (AFRL), under contracts
FA8750-10-C-0237 (``CTSRD'') and HR0011-18-C-0016 (``ECATS'').
The views, opinions, and/or findings contained in this report are those of the authors and should not be interpreted as representing the official views or policies of the Department of Defense or the U.S. Government.
This work was supported in part by the Innovate UK project Digital Security by
Design (DSbD) Technology Platform Prototype, 105694.
We also acknowledge the EPSRC REMS Programme Grant (EP/K008528/1), the
ERC ELVER Advanced Grant (789108), Arm Limited,
HP Enterprise, and Google, Inc.
Approved for Public Release, Distribution Unlimited.

\bibliographystyle{abbrv}
\bibliography{cheri}

\end{document}
